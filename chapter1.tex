% 导入配置
% \documentclass[lang = cn, scheme = chinese, thmcnt = section]{elegantbook}
% elegantbook      设置elegantbook文档类
% lang = cn        设置中文环境
% scheme = chinese 设置标题为中文
% thmcnt = section 设置计数器


%% 1.封面设置

\title{Algebra Chapter 0 - Paolo Aluffi NoteBook}                % 文档标题

\author{若水}               % 作者

\extrainfo{上善若水任方圆}   % 箴言

\date{\today}               % 日期

\logo{PiCreatures_happy.pdf}        % 设置Logo

\cover{阿基米德螺旋曲线.pdf}          % 设置封面图片

% 修改标题页的色带
\definecolor{customcolor}{RGB}{135, 206, 250} 
% 定义一个名为customcolor的颜色,RGB颜色值为(135, 206, 250)

\colorlet{coverlinecolor}{customcolor}     % 将coverlinecolor颜色设置为customcolor颜色

%% 2.目录设置
\setcounter{tocdepth}{3}  % 目录深度为3

%% 3.引入宏包
\usepackage[all]{xy}
\usepackage{bbm, svg, graphicx, float, extpfeil, amsmath, amssymb, mathrsfs, mathalpha, boondox-cal, boondox-calo, hyperref}


%% 4.定义命令
\newcommand{\N}{\mathbb{N}}            % 自然数集合
\newcommand{\R}{\mathbb{R}}            % 实数集合
\newcommand{\C}{\mathbb{C}}  		   % 复数集合
\newcommand{\Q}{\mathbb{Q}}            % 有理数集合
\newcommand{\Z}{\mathbb{Z}}            % 整数集合
\newcommand{\sub}{\subset}             % 包含
\newcommand{\im}{\text{im }}           % 像
\newcommand{\lang}{\langle}            % 左尖括号
\newcommand{\rang}{\rangle}            % 右尖括号
\newcommand{\function}[5]{
	\begin{align*}
		#1:\begin{aligned}[t]
			#2 &\longrightarrow #3\\
			#4 &\longmapsto #5
		\end{aligned}
	\end{align*}
}                                     % 函数

\newcommand{\lhdneq}{%
	\mathrel{\ooalign{$\lneq$\cr\raise.22ex\hbox{$\lhd$}\cr}}} % 真正规子群

\newcommand{\rhdneq}{%
	\mathrel{\ooalign{$\gneq$\cr\raise.22ex\hbox{$\rhd$}\cr}}} % 真正规子群

% \begin{document}
	
\chapter{集合论与范畴论}

\section{朴素集合论}

\subsection{集合}

\begin{theorem}{Russell悖论}
	\href{https://en.wikipedia.org/wiki/Russell%27s_paradox}{Russell悖论}:对于集合$R=\{r:r\notin r\}$,成立
	$$
	R\in R\iff R\notin R
	$$
\end{theorem}

\subsection{集合的关系}

\begin{itemize}
	\item 集合:$\varnothing,\N,\Z,\Q,\R,\C$
	\item 量词:$\exists,\forall,\exists!$
	\item 包含:$\sub,\subseteq,\subsetneq$
	\item 势:$|S|$
	\item 幂集(power set):$\mathscr{P}(S)=\{T\sub S\}$
\end{itemize}

\subsection{集合的运算}

\begin{itemize}
	\item 运算:$\cup,\cap,\setminus,\sqcup,\times$
\end{itemize}

\subsection{不交并与积}

\begin{definition}{不交并 disjoint union}
	$$
	A\sqcup B=(\{0\}\times A)\times(\{1\}\times B)
	$$
\end{definition}

\begin{definition}{积 product}
	$$
	A\times B=\{ (a,b):a\in A,b\in B \}
	$$
\end{definition}

\subsection{等价关系,划分与商}

\begin{definition}{关系 relation}
	定义集合$S$上的一个关系为$R\sub S\times S$,称$a$和$b$存在关系$R$,并记做记作$aRb$,如果$(a,b)\in R$。
\end{definition}

\begin{definition}{等价关系 equivalence relation}
	称集合$S$上的关系$\sim $为等价关系,如果其满足如下性质。
	\begin{enumerate}
		\item 自反性(reflexivity):$a\sim a$
		\item 对称性(symmetry):$a\sim b\implies b\sim a$
		\item 传递性(transitivity):$a\sim b,b\sim c\implies a\sim c$
	\end{enumerate}
\end{definition}

\begin{definition}{等价类 equivalence class}
	对于$a\in S$,定义$a$关于等价关系$\sim$的等价类为
	$$
	[a]_\sim=\{ s\in S:a\sim s \}
	$$
	
	对于$A\sub S$,定义$A$关于等价关系$\sim$的等价类为
	$$
	[A]_\sim=\{ s\in S:\exists a\in A,\text{ s.t. }a\sim s \}
	$$
\end{definition}

\begin{example}
	好朋友:满足自反性和对称性,但不满足传递性。
\end{example}

\begin{proposition}{等价类的性质}
	$$
	\left[ \bigcup_{\lambda\in\Lambda}S_\lambda \right]_\sim
	=\bigcup_{\lambda\in\Lambda}[S_\lambda]_\sim=\left[ \bigcup_{\lambda\in\Lambda}[S_\lambda]_\sim \right]_\sim,\qquad
	\left[ \bigcap_{\lambda\in\Lambda}S_\lambda \right]_\sim
	\sub \bigcap_{\lambda\in\Lambda}[S_\lambda]_\sim=\left[ \bigcap_{\lambda\in\Lambda}[S_\lambda]_\sim \right]_\sim
	$$
\end{proposition}

\begin{definition}{划分 partition}
	称子集族$\mathscr{P}\sub\mathscr{P}(S)$为集合$S$的一个划分,如果
	$$
	\bigsqcup_{T\in\mathscr{P}}T=S
	$$
	
	定义集合$S$关于等价关系$\sim$的划分为
	$$
	\mathscr{P}_\sim=\{ [a]_\sim:a\in S \}
	$$
\end{definition}

\begin{definition}{商 quotient}
	定义集合$S$关于等价关系$\sim$的商为
	$$
	S/\sim=\{ [a]_\sim:a\in S \}
	$$
\end{definition}

\begin{example}
	对于实数域$\R$,定义等价关系$\sim$:
	$$
	a\sim b\iff a-b\in\Z
	$$
	对于实数平面$\R^2$,定义等价关系$\approx$:
	$$
	(a_1,a_2)\approx(b_1,b_2)\iff
	a_1-b_1\in\Z,a_2-b_2\in\Z
	$$
	那么
	$$
	\R/\sim=[0,1),\qquad \R^2/\approx [0,1)\times[0,1)
	$$
\end{example}

\begin{theorem}{等价类的不交性}{等价关系的不交性}
	如果$\sim$为集合$S$上的等价关系,那么对于任意$a,b\in S$,或$[a]_\sim=[b]_\sim$成立,或$[a]_\sim\cap [b]_\sim=\varnothing$成立,且仅成立其中之一。
\end{theorem}

\begin{proof}
	第一,显然二者不能同时成立。事实上,如果$[a]_\sim=[b]_\sim$且$[a]_\sim\cap [b]_\sim=\varnothing$,那么$[a]_\sim=[b]_\sim=\varnothing$,但是$a\in[a]_\sim$,矛盾!
	
	第二,如果$[a]_\sim=[b]_\sim$,那么命题得证!如果$[a]_\sim\ne[b]_\sim$,我们来证明$[a]_\sim\cap [b]_\sim=\varnothing$。反证,假设存在$c\in S$,使得$c\in [a]_\sim\cap[b]_\sim$,那么$c\in [a]_\sim$且$c\in [b]_\sim$。任取$m\in[a]_\sim$,由于$c\in [a]_\sim$,那么$c\sim m$,又$c\in [b]_\sim$,那么$m\in [b]_\sim$,于是$[a]_\sim\sub[b]_\sim$,同理$[b]_\sim\sub[a]_\sim$,进而$[a]_\sim=[b]_\sim$,矛盾!因此$[a]_\sim\cap [b]_\sim=\varnothing$。
\end{proof}

\begin{theorem}{划分$\iff$等价关系}{1.1.2}
	\begin{enumerate}
		\item 如果子集族$\mathscr{P}\sub \mathscr{P}(S)$为集合$S$上的划分,那么存在$S$上的等价关系$\sim$,使得$\mathscr{P}=\mathscr{P}_\sim$。
		\item 如果关系$\sim$为集合$S$上的等价关系,那么子集族$\mathscr{P}_\sim$为$S$上的一个划分。
	\end{enumerate}
\end{theorem}

\begin{proof}
	对于1,如果子集族$\mathscr{P}\sub \mathscr{P}(S)$为集合$S$上的划分,那么
	$$
	\bigsqcup_{ T\in\mathscr{P}}T_\lambda=S
	$$
	定义$S$上的一个等价关系:
	$$
	a\sim b\iff 
	\exists T\in\mathscr{P},a,b\in T
	$$
	
	事实上,对于任意$a\in S$,由于$\displaystyle\bigcup_{ T\in\mathscr{P}}T_\lambda=S$,那么存在$T\in\mathscr{P}$,使得成立$a\in T$,自反性得证!
	
	对于任意$a,b\in S$,如果$a\sim b$,那么存在$T\in\mathscr{P}$,使得成立$a,b\in T$,显然$b\sim a$,对称性得证!
	
	对于任意$a,b,c\in S$,如果$a\sim b,b\sim c$,那么存在$T_1,T_2\in\mathscr{P}$,使得成立$a,b\in T_1,b,c\in T_2$,于是$b\in T_1\cap T_2$。而$\mathscr{P}$中元素的不交性保证了$T_1=T_2$,进而$a,c\in T_1=T_2$,即$a\sim c$,传递性得证!
	
	于是$\sim$为$S$上的等价关系。由$\sim$的定义,不难发现对于任意$T\in\mathscr{P}$,任取$t\in T\sub S$,那么$T=[t]_\sim$,这是因为
	$$
	T=[t]_\sim\iff
	T=\{ s\in S:s\sim t \}\iff
	T=\{ s\in S:\exists R\in\mathscr{P},s,t\in R\}\iff
	T=\{ s\in T \}
	$$
	
	下面我们依托这个事实来证明:$\mathscr{P}=\mathscr{P}_\sim$。
	
	一方面,任取$T\in\mathscr{P}$,那么任取$t\in T\sub S$,可得$T=[t]_\sim \in\mathscr{P}_\sim$,于是$\mathscr{P}\sub\mathscr{P}_\sim$。
	
	另一方面,任取$[t]_\sim\in\mathscr{P}_\sim$,由于$\displaystyle\bigcup_{ T\in\mathscr{P}}T_\lambda=S\ni  t$,那么存在$T\in\mathscr{P}$,使得成立$t\in T$,于是$[t]_\sim =T\in \mathscr{P}$,于是$\mathscr{P}\supset\mathscr{P}_\sim$。
	
	综合两方面,$\mathscr{P}=\mathscr{P}_\sim$成立,原命题得证!
	
	对于2,如果关系$\sim$为集合$S$上的等价关系,那么$\mathscr{P}_\sim$中元素的不交性由定理\ref{thm:等价关系的不交性}保证。我们来证明
	$$
	\bigcup_{T\in\mathscr{P}_\sim}T=S
	\iff\bigcup_{a\in S}[a]_{\sim}=S
	$$
	
	一方面,由于$[a]_\sim\sub S$,于是$\displaystyle\bigcup_{a\in S}[a]_{\sim}\sub S$。
	
	另一方面,由于对于任意$a\in S$,显然$a\in[a]_\sim$,于是$\displaystyle\bigcup_{a\in S}[a]_{\sim}\supset S$。综合两方面,$\displaystyle\bigcup_{a\in S}[a]_{\sim}=S$。
	
	综上所述,原命题得证!
\end{proof}

\begin{definition}{第二类Stirling数}
	称将$n$个两两不同的元素划分为$k$个互不区分的非空子集的方案数称为第二类Stirling数,记作$S(n,k)$。递推公式如下
	$$
	S(n+1,k)=S(n,k-1)+kS(n,k)
	$$
	通项公式如下:
	$$
	S(n,k)=\frac{1}{k!}\sum_{i=0}^{k}(-1)^k{k\choose i}(k-i)^n
	$$
\end{definition}

\begin{definition}{Bell数}
	称将$n$个两两不同的元素划分为互不区分的非空子集的方案数为第$n$个Bell数,记作$B(n)$。解析表达式为
	$$
	B(n)=\sum_{k=0}^{n}S(n,k)=\sum_{k=0}^{n}\sum_{i=0}^{k}(-1)^k\frac{1}{k!}{k\choose i}(k-i)^n
	$$
\end{definition}

\begin{proposition}
	$n$个元素的集合上可以定义$B(n)$个不同的等价关系,其中$B(n)$为第$n$个Bell数。
\end{proposition}

\begin{table}[h]
	\centering
	\caption{第二类Stirling数与Bell数表}
	\resizebox{\textwidth}{!}{%
		\begin{tabular}{|c|c|c|c|c|c|c|c|c|c|c|c|c|}
			\hline
			$n$ & $S(n,0)$ & $S(n,1)$ & $S(n,2)$ & $S(n,3)$ & $S(n,4)$ & $S(n,5)$ & $S(n,6)$ & $S(n,7)$ & $S(n,8)$ & $S(n,9)$ & $S(n,10)$ & $B(n)$ \\
			\hline
			$0$ & $1$ & $0$ & $0$ & $0$ & $0$ & $0$ & $0$ & $0$ & $0$ & $0$ & $0$ & $1$ \\
			\hline
			$1$ & $0$ & $1$ & $0$ & $0$ & $0$ & $0$ & $0$ & $0$ & $0$ & $0$ & $0$ & $1$ \\
			\hline
			$2$ & $0$ & $1$ & $1$ & $0$ & $0$ & $0$ & $0$ & $0$ & $0$ & $0$ & $0$ & $2$ \\
			\hline
			$3$ & $0$ & $1$ & $3$ & $1$ & $0$ & $0$ & $0$ & $0$ & $0$ & $0$ & $0$ & $5$ \\
			\hline
			$4$ & $0$ & $1$ & $7$ & $6$ & $1$ & $0$ & $0$ & $0$ & $0$ & $0$ & $0$ & $15$ \\
			\hline
			$5$ & $0$ & $1$ & $15$ & $25$ & $10$ & $1$ & $0$ & $0$ & $0$ & $0$ & $0$ & $52$ \\
			\hline
			$6$ & $0$ & $1$ & $31$ & $90$ & $65$ & $15$ & $1$ & $0$ & $0$ & $0$ & $0$ & $203$ \\
			\hline
			$7$ & $0$ & $1$ & $63$ & $301$ & $350$ & $140$ & $21$ & $1$ & $0$ & $0$ & $0$ & $877$ \\
			\hline
			$8$ & $0$ & $1$ & $127$ & $966$ & $1701$ & $1050$ & $266$ & $28$ & $1$ & $0$ & $0$ & $4140$ \\
			\hline
			$9$ & $0$ & $1$ & $255$ & $3025$ & $7770$ & $6951$ & $2646$ & $462$ & $36$ & $1$ & $0$ & $21147$ \\
			\hline
			$10$ & $0$ & $1$ & $511$ & $9330$ & $34105$ & $42525$ & $22827$ & $5880$ & $750$ & $45$ & $1$ & $115975$ \\
			\hline
		\end{tabular}%
	}
\end{table}

\section{集合间的函数}

\subsection{定义}

\begin{definition}{函数 function}
	称$f$为定义域为$A$,陪域为$B$的函数,如果对于任意$a\in A$,存在且存在唯一$b\in B$,使得成立$f(a)=b$。记作
	\begin{align*}
		f: \begin{aligned}[t]
			A&\longrightarrow B\\
			a&\longmapsto f(a)
		\end{aligned}
	\end{align*}
	或者以图(diagram)表示
	$$
	\xymatrix{
		A \ar^f[r] &B
	}
	$$
\end{definition}

\begin{definition}{图像 graph}
	对于函数$f:A\to B$,定义其图像为
	$$
	\Gamma_f=\{ (a,b):a\in A,b=f(a) \}\sub A\times B
	$$
\end{definition}

\begin{proposition}
	对于函数$f:A\to B$,成立$\Gamma_f\cong A$。
\end{proposition}

\begin{proof}
	定义函数
	\function{\varphi}{A}{\Gamma_f}{a}{(a,f(a))}
	显然$\varphi$为双射。
\end{proof}

\begin{definition}{单位函数 identity function}
	\begin{align*}
		\mathbbm{1}_A:\begin{aligned}[t]
			A&\longrightarrow A\\
			a&\longmapsto a
		\end{aligned}
	\end{align*}
\end{definition}

\begin{definition}
	定义$A\to B$上的所有函数构成集合$B^A=\{ f:A\to B \}$。
\end{definition}

\begin{definition}{函数在子集上的限制}
	对于函数$f:A\to B$,以及$S\sub A$,定义
	\begin{align*}
		f|_S:\begin{aligned}[t]
			S&\longrightarrow B\\
			s&\longmapsto f(s)
		\end{aligned}
	\end{align*}
\end{definition}

\begin{definition}{像 image}
	对于函数$f:A\to B$,定义子集$S\subset A$的像为
	$$
	f(S)=\{ f(s):s\in S \}
	$$
\end{definition}

\begin{definition}{原像 preimage}
	对于函数$f:A\to B$,定义子集$T\subset B$的原像为
	$$
	f^{-1}(T)=\{ a\in A:f(a)\in T \}
	$$
\end{definition}

\begin{proposition}{像与原像的性质}
	对于函数$f:X\to Y$,如下命题成立。
	\begin{enumerate}
		\item 如果$A\sub B\sub X$,那么
		$$
		f(A)\sub f(B)
		$$
		\item 如果$A\sub B\sub Y$,那么
		$$
		f^{-1}(A)\sub f^{-1}(B)
		$$
		\item 如果$B\sub Y$,那么
		$$
		f^{-1}(B^c)=(f^{-1}(B))^c
		$$
		\item 如果$B,D\sub Y$,那么
		$$
		f^{-1}(B\setminus D)=f^{-1}(B)\setminus f^{-1}(D)
		$$
		\item 如果$\{ B_\lambda \}_{\lambda\in\Lambda}\sub\mathscr{P}(Y)$,那么
		$$
		f^{-1}\left( \bigcup_{\lambda\in\Lambda}B_\lambda \right)=\bigcup_{\lambda\in\Lambda}f^{-1}(B_\lambda),\qquad 
		f^{-1}\left( \bigcap_{\lambda\in\Lambda}B_\lambda \right)=\bigcap_{\lambda\in\Lambda}f^{-1}(B_\lambda)
		$$
		\item 如果$\{ A_\lambda \}_{\lambda\in\Lambda}\sub\mathscr{P}(X)$,那么
		$$
		f\left( \bigcup_{\lambda\in\Lambda}A_\lambda \right)=\bigcup_{\lambda\in\Lambda}f(A_\lambda),\qquad 
		f\left( \bigcap_{\lambda\in\Lambda}[A_\lambda]_\sim \right)=\bigcap_{\lambda\in\Lambda}f(A_\lambda)
		$$
		\item 如果$B\sub Y$,那么
		$$
		f(f^{-1}(B))=B\cap\text{im }f
		$$
		\item 定义等价关系$x\sim y\iff f(x)=f(y)$。如果$A\sub X$,那么
		$$
		f^{-1}(f(A))=[A]_\sim
		$$
	\end{enumerate}
\end{proposition}

\begin{proposition}
	对于集合$S$,定义$2^S=\{ f:S\to \{0,1\} \}$,那么$2^S\cong \mathscr{P}(S)$。
\end{proposition}

\begin{proof}
	定义函数
	\function{\varphi}{2^S}{\mathscr{P}(S)}{f}{\{ s\in S:f(s)=1 \}}
	下面我们证明$\varphi$为双射。
	
	对于单射性,任取$f_1,f_2\sub 2^S$,满足$\varphi(f_1)=\varphi(f_2)$,即$\{ s\in S:f_1(s)=1 \}=\{ s\in S:f_2(s)=1 \}$,那么$f_1=f_2$,因此$\varphi$为单射。
	
	对于满射性,任取$A\sub S$,令
	$$
	f(s)=\begin{cases}
		1,\quad & s\in A\\
		0,\quad & s\in S\setminus A
	\end{cases}
	$$
	显然$\varphi(f)=A$,因此$\varphi$为满射。
	
	综上所述,$2^S\cong \mathscr{P}(S)$,原命题得证!
\end{proof}

\subsection{多重集合与指标集}

\begin{definition}{多重集合 multiset}
	定义元素可重复的集合为多重集合。事实上,多重集合可以函数的观点考虑,例如对于多重集合$M$,其中$m_k$出现了$n_k$次,那么$M$即与如下函数同构。
	\begin{align*}
		m:\begin{aligned}[t]
			\{m_k:k\in\N^*\}&\longrightarrow\N^*\\
			m_k&\longmapsto n_k
		\end{aligned}
	\end{align*}
\end{definition}

\begin{definition}{指标集 indexed set}
	数列$\{ a_n \}_{n=1}^{\infty}$的本质为如下函数。
	\begin{align*}
		a:\begin{aligned}[t]
			\N^*&\longrightarrow\C\\
			n&\longmapsto a_n
		\end{aligned}
	\end{align*}
\end{definition}

\subsection{复合函数}

\begin{definition}{复合函数 composition of function}
	对于函数$f:A\to B$和$g:B\to C$,定义其复合函数为
	\begin{align*}
		f\circ g:\begin{aligned}[t]
			A&\longrightarrow C\\
			a&\longmapsto g(f(a))
		\end{aligned}
	\end{align*}
	交换图如下
	$$
	\xymatrix{
		A \ar@/_1pc/[rr]_{g\circ f} \ar^f[r] &B \ar^g[r] &C
	}\qquad \text{或}\qquad
	\xymatrix{
		A \ar^f[r]\ar_{g\circ f}[dr] &B \ar^g[d]\\
		& C
	}
	$$
\end{definition}

\subsection{单射,满射与双射}

\begin{definition}{单射 injection}
	称函数$f:A\to B$是单的,如果$f(a_1)=f(a_2)$,那么$a_1=a_2$。单射记作$f:A\hookrightarrow B$。
\end{definition}

\begin{definition}{满射 surjection}
	称函数$f:A\to B$是满的,如果对于任意$b\in B$,存在$a\in A$,使得成立$f(a)=b$。满射记作$f:A\twoheadrightarrow B$。
\end{definition}

\begin{definition}{双射 bijection}
	称函数$f:A\to B$是双射,并记作$f:A\xrightarrow{\sim}B$,如果其既单又满。
\end{definition}

\begin{example}
	对于有$n$个元素的集合$S$,存在$n!$个双射。
\end{example}

\begin{proposition}{双射的复合为双射}{1.2.4复合}
	双射的复合为双射。
\end{proposition}

\begin{proof}
	如果函数$f,g$均为双射,那么其逆函数$f^{-1},g^{-1}$满足
	$$
	f\circ f^{-1}=f^{-1}\circ f=\mathbbm{1},\qquad 
	g\circ g^{-1}=g^{-1}\circ g=\mathbbm{1}
	$$
	于是
	$$
	(f\circ g)(g^{-1}\circ f^{-1})=(g^{-1}\circ f^{-1})(f\circ g)=\mathbbm{1}
	$$
	由定理\ref{thm:1.2.2}与定理\ref{thm:1.2.3},复合$f\circ g$为双射。
\end{proof}

\begin{proposition}{双射的逆为双射}{1.2.4逆}
	双射的逆为双射。
\end{proposition}

\begin{proof}
	如果函数$f$为双射,其逆函数$f^{-1}$满足
	$$
	f\circ f^{-1}=f^{-1}\circ f=\mathbbm{1}
	$$
	由定理\ref{thm:1.2.2}与定理\ref{thm:1.2.3},逆$f^{-1}$为双射。
\end{proof}

\begin{definition}{同构的 isomorphic}
	称集合$A$与$B$为同构的,并记做$A\cong B$,如果存在双射$f:A\to B$。
\end{definition}

\begin{proposition}{同构为等价关系}
	同构为等价关系。
\end{proposition}

\begin{proof}
	定义关系
	$$
	A\sim B \iff A\cong B \iff \exists\text{双射}f:A\to B
	$$
	
	对于反身性,显然$\mathbbm{1}_A$为双射,因此$A\sim A$。
	
	对于对称性,如果$A\sim B$,那么存在双射$f:A\to B$,由命题\ref{pro:1.2.4逆},$f^{-1}:B\to A$为双射,因此$B\sim A$。
	
	对于传递性,如果$A\sim B,B\sim C$,那么存在双射$f:A\to B,g:B\to C$,由命题\ref{pro:1.2.4复合},$f\circ g:A\to C$为双射,因此$A\sim C$。
	
	因此,同构为等价关系。
\end{proof}

\begin{proposition}
	如果$A_1\cong A_2,B_1\cong B_2$,且$A_1\cap B_1=A_2\cap B_2=\varnothing$,那么$A_1\cup B_1\cong A_2\cup B_2$。
\end{proposition}

\begin{proof}
	由于$A_1\cong A_2,B_1\cong B_2$,那么存在双射$f:A_1\to A_2$和$g:B_1\to B_2$,定义函数
	\begin{align*}
		h:\begin{aligned}[t]
			A_1\cup B_1&\longrightarrow A_2\cup B_2\\
			x&\longmapsto 
			\begin{cases}
				f(x),\quad & x\in A_1\\
				g(x),\quad & x\in B_1
			\end{cases}
		\end{aligned}
	\end{align*}
	由$A_1\cap B_1=A_2\cap B_2=\varnothing$,容易验证$h$为双射,因此$A_1\cup B_1\cong A_2\cup B_2$。
\end{proof}

\begin{definition}{逆 inverse}
	对于双射$f:A\to B$,定义其逆为
	\begin{align*}
		f^{-1}:\begin{aligned}[t]
			B&\longrightarrow A\\
			f(a)&\longmapsto a
		\end{aligned}
	\end{align*}
	交换图如下
	$$
	\xymatrix{
		A \ar@/_1pc/[rr]_{\mathbbm{1}_A} \ar^f[r] &B \ar^g[r] &A
	},\qquad 
	\xymatrix{
		B \ar@/_1pc/[rr]_{\mathbbm{1}_B} \ar^f[r] &A \ar^g[r] &B
	}
	$$
\end{definition}

\begin{definition}{左逆 left-inverse}
	称函数$g:\im f\to A$为函数$f:A\to B$的左逆,如果成立$g\circ f=\mathbbm{1}_A$。
\end{definition}

\begin{definition}{右逆 right-inverse}
	称函数$g:B\to A$为函数$f:A\to B$的右逆,如果成立$f\circ g=\mathbbm{1}_B$。
\end{definition}

\begin{proposition}{双射$\iff$存在左右逆}{双射的等价条件}
	双射$\iff$存在左逆和右逆。
\end{proposition}

\begin{proof}
	见定理\ref{thm:1.2.2}与定理\ref{thm:1.2.3}。
\end{proof}

\subsection{单态射与满态射}

\begin{definition}{单态射 monomorphism}
	称函数$f:A\to B$是单态射,如果对于任意集合$Z$,以及任意函数$\alpha_1,\alpha_2:Z\to A$,成立
	$$
	f\circ \alpha_1=f\circ \alpha_2\implies\alpha_1=\alpha_2
	$$
\end{definition}

\begin{definition}{满态射 epimorphism}
	称函数$f:A\to B$是满态射,如果对于任意集合$Z$,以及任意函数$\beta_1,\beta_2:B\to Z$,成立
	$$
	\beta_1\circ f=\beta_2\circ f\implies \beta_1=\beta_2
	$$
\end{definition}

\begin{theorem}{}{1.2.2}
	对于函数$f:A\to B$,如下命题等价。
	\begin{enumerate}
		\item $f$为单射。
		\item $f$存在左逆。
		\item $f$为单态射。
	\end{enumerate}
\end{theorem}

\begin{proof}
	1$\implies$2:如果$f$为单射,定义函数
	\function{g}{\text{im }f}{A}{f(a)}{a}
	首先来验证$g$的定义是良好的。取$a_1,a_2\in A$,满足$f(a_1)=f(a_2)$,由$f$的单射性,$a_1=a_2$,于是$g$定义良好。其次来验证$g\circ f=\mathbbm{1}$。任取$a\in A$,注意到$(g\circ f)(a)=g(f(a))=a$,那么$g\circ f=\mathbbm{1}$。综合这两点,$f$存在左逆$g$。
	
	1$\implies$3:如果$f$为单射,任取$\alpha_1,\alpha_2:Z\to A$,满足$f\circ \alpha_1=f\circ \alpha_2$。任取$z\in Z$,注意到
	$$
	f(\alpha_1(z))=(f\circ \alpha_1)(z)=(f\circ \alpha_2)(z)=f(\alpha_2(z))
	$$
	于是$\alpha_1=\alpha_2$,进而$f$是单态射。
	
	2$\implies $3:如果$f$存在左逆$g$,任取$\alpha_1,\alpha_2$,满足$f\circ \alpha_1=f\circ \alpha_2$,那么
	$$
	\alpha_1=\mathbbm{1}\circ \alpha_1=g\circ f\circ \alpha_1=g\circ f\circ \alpha_2=\mathbbm{1}\circ \alpha_2=\alpha_2
	$$
	于是$f$是单态射。
	
	2$\implies $1:如果$f$存在左逆$g$,任取$a_1,a_2\in A$,满足$f(a_1)=f(a_2)$,那么
	$$
	a_1=\mathbbm{1}(a_1)=(g\circ f)(a_1)=g(f(a_1))=g(f(a_2))=(g\circ f)(a_2)=\mathbbm{1}(a_2)=a_2
	$$
	于是$f$是单射。
	
	3$\implies$1:如果$f$是单态射,任取$a_1,a_2\in A$,满足$f(a_1)=f(a_2)$。定义$\alpha_1:Z\to \{a_1\}$和$\alpha_2:Z\to \{a_2\}$,任取$z\in Z$,注意到
	$$
	(f\circ \alpha_1)(z)=f(\alpha_1(z))=f(a_1)=f(a_2)=f(\alpha_2(z))=(f\circ \alpha_2)(z)
	$$
	因此$f\circ\alpha_1=f\circ\alpha_2$,于是$\alpha_1=\alpha_2$,即$a_1=a_2$,进而$f$是单射。
\end{proof}

\begin{theorem}{}{1.2.3}
	对于函数$f:A\to B$,如下命题等价。
	\begin{enumerate}
		\item $f$为满射。
		\item $f$存在右逆。
		\item $f$为满态射。
	\end{enumerate}
\end{theorem}

\begin{proof}
	1$\implies$2:如果$f$是满射,定义函数
	\function{g}{B}{A}{b}{a}
	这里要说明的是对于特别的$b\in B$,$f^{-1}(b)$中的元素可能不唯一,这时候任取其一即可,此时便说明$g$定义良好。然后我们来验证$f\circ g=\mathbbm{1}$。任取$b\in B$,注意到$(f\circ g)(b)=f(g(b))=f(f^{-1}(b))=b$,那么$f\circ g=\mathbbm{1}$,进而$f$存在右逆$g$。
	
	1$\implies$3:如果$f$为满射,任取$\beta_1,\beta_2$,满足$\beta_1\circ f=\beta_2\circ f$。任取$b\in B$,存在$a\in A$,使得成立$f(a)=b$,因此
	$$
	\beta_1(b)=\beta_1(f(a))=(\beta_1\circ f)(a)=(\beta_2\circ f)(a)=\beta_2(f(a))=\beta_2(b)
	$$
	于是$\beta_1=\beta_2$,进而$f$是满态射。
	
	2$\implies $3:如果$f$存在右逆$g$,任取$\beta_1,\beta_2$,满足$\beta_1\circ f=\beta_2\circ f$,那么
	$$
	\beta_1=\beta_1\circ \mathbbm{1}=\beta_1\circ f\circ g=\beta_2\circ f\circ g=\beta_2\circ \mathbbm{1}=\beta_2
	$$
	于是$f$是满态射。
	
	2$\implies$1:如果$f$存在右逆$g$,任取$b\in B$,注意到$g(b)\in A$,且$f(g(b))=(f\circ g)(b)=\mathbbm{1}(b)=b$,因此$f$为满射。
	
	3$\implies$1:如果$f$是满态射,任取$b\in B$。定义$\beta_1:B\to\{1\}$以及
	\begin{align*}
		\beta_2: \begin{aligned}[t]
			B &\longrightarrow \{0,1\}\\
			b &\longmapsto \begin{cases}
				1,\quad & b\in\mathrm{im}f\\
				0,\quad & b\in B\setminus\mathrm{im}f
			\end{cases}
		\end{aligned}
	\end{align*}
	任取$a\in A$,注意到
	$$
	(\beta_1\circ f)(a)=\beta_1(f(a))=1=\beta_2(f(a))=(\beta_2\circ f)(a)
	$$
	因此$\beta_1\circ f=\beta_2\circ f$,于是$\beta_1=\beta_2$,即$b\in\mathrm{im}f$,进而$f$是满射。
\end{proof}

\subsection{自然投影}

\begin{definition}{自然投影 natural projection}
	对于集合$A$和$B$,定义其自然投影为
	\begin{align*}
		\pi_A: \begin{aligned}[t]
			A\times B&\longrightarrow A\\
			(a,b)&\longmapsto a
		\end{aligned}
	\end{align*}
	\begin{align*}
		\pi_B: \begin{aligned}[t]
			A\times B\longrightarrow B\\
			(a,b)\longmapsto B
		\end{aligned}
	\end{align*}
	交换图如下
	$$
	\xymatrix{
		& A\times B\ar@{^{}->>}[ld]_{\pi_A} \ar@{->>}[rd]^{\pi_B} &\\
		A &           & B
	}
	$$
\end{definition}

\begin{example}
	解释函数$f:A\to B$如何决定$\pi_A$的右逆。
	
	由于
	\function{\pi_A}{A\times B}{A}{(a,b)}{a}
	定义
	\function{\varphi}{A}{A\times B}{a}{(a,f(a))}
	显然$\pi_A\circ \varphi=\mathbbm{1}_A$,且$g$由$f$决定。交换图如下
	$$
	\xymatrix@=11ex{
		A \ar@/_2pc/[rr]_{\mathbbm{1}_A} \ar@{^{(}->}[r]^{\varphi}_{a\mapsto (a,f(a))} & A\times B \ar@{->>}[r]^{\pi_A}_{(a,b)\mapsto a} & A
	}
	$$
\end{example}

\subsection{正则分解}

\begin{definition}{正则分解 canonical decomposition}
	对于函数$f:A\to B$,以及等价关系$a_1\sim a_2\iff f(a_1)=f(a_2)$,$f$可作如下正则分解。
	$$
	\xymatrix@=9ex{
		A \ar@/^2pc/[rrr]^{f} \ar@{->>}[r]_{a\mapsto[a]_{\sim}} & A/\sim \ar@{->}[r]_{[a]_\sim\mapsto f(a)}^{\sim} & \mathrm{im}f \ar@{^{(}->}[r]_{f(a)\mapsto f(a)} & B
	}
	$$
\end{definition}

\begin{example}
	描述如下函数的正则分解:
	\function{\varphi}{\R}{\C}{r}{\mathrm{e}^{2\pi i r}}
	定义等价关系
	$$
	r_1\sim r_2 \iff \varphi(r_1)=\varphi(r_2) \iff r_1-r_2\in\Z
	$$
	那么$\varphi$的正则分解为
	$$
	\xymatrix@=10ex{
		\R \ar@/^2pc/[rrr]^{\varphi} \ar@{->>}[r]_(0.45){r\mapsto(r)} & [0,1) \ar@{->}[r]_(0.4){(r)\mapsto \mathrm{e}^{2\pi i r}}^(0.4){\sim} & \partial\mathbb{D} \ar@{^{(}->}[r]_(0.625){\mathrm{e}^{2\pi i r}\mapsto \mathrm{e}^{2\pi i r}} & \C
	}
	$$
\end{example}

\begin{example}
	取$f:\R\to\R$为$x\mapsto x^2$,等价关系$x\sim y\iff |x|=|y|$,那么$f$​可做如下正则分解。
	$$
	\xymatrix@=9ex{
		\R \ar@/^2pc/[rrr]^{f} \ar@{->>}[r]_{x\mapsto|x|} & \R^2\cup\{0\} \ar@{->}[r]_{|x|\mapsto x^2}^{\sim} & \R^2\cup\{0\} \ar@{^{(}->}[r]_{x^2\mapsto x^2} & \R
	}
	$$
\end{example}

\section{范畴论}

\subsection{定义}

\begin{definition}{范畴 category}
	一个范畴$\mathsf{C}$包含如下两类内容。
	\begin{enumerate}
		\item 对象(object):$\mathrm{Obj}(\mathsf{C})$
		\item 态射(morphism):$\mathrm{Hom}_\mathsf{C}(A,B)$,满足如下公理。其中称$A$为源(source),$B$为目标(target)。
		\begin{itemize}
			\item 恒等态射:对于任意对象$S\in\mathrm{Obj}(\mathsf{C})$,存在恒等态射$\mathbbm{1}_S\in\mathrm{Hom}_\mathsf{C}(S,S)$。同时,对于任意态射$f\in \mathrm{Hom}_\mathsf{C}(A,B)$,成立$f\circ\mathbbm{1}_A=f$和$\mathbbm{1}_B\circ f=f$。
			\item 态射复合:对于任意对象$A,B,C\in\mathrm{Obj}(\mathsf{C})$,存在态射复合映射
			
		\begin{align*}
				\circ :\begin{aligned}[t]
					\mathrm{Hom}_\mathsf{C}(A,B)\times \mathrm{Hom}_\mathsf{C}(B,C)&\longrightarrow \mathrm{Hom}_\mathsf{C}(A,C)\\
					(f,g)&\longmapsto g\circ f
				\end{aligned}
			\end{align*}
			\item 结合律:对于任意态射$f\in \mathrm{Hom}_\mathsf{C}(A,B),g\in \mathrm{Hom}_\mathsf{C}(B,C),h\in \mathrm{Hom}_\mathsf{C}(C,D)$,成立$(h\circ g)\circ f=h\circ (g\circ f)$。
		\end{itemize}
	\end{enumerate}
\end{definition}

\begin{definition}{不交性}
	称态射$\mathrm{Hom}_\mathsf{C}(A,B)$和$\mathrm{Hom}_\mathsf{C}(C,D)$是不相交的,如果$A\ne C$或$B\ne D$。
\end{definition}

\begin{definition}{自态射 endormorphism}
	对于范畴$\mathsf{C}$,称态射$\mathrm{End}_\mathsf{C}(S)=\mathrm{Hom}_\mathsf{C}(S,S)$为对象$S\in\mathrm{Obj}(\mathsf{C})$的自态射。注意到,$\mathbbm{1}_S\in \mathrm{End}_\mathsf{C}(S)$,且如果$f,g\in\mathrm{End}_\mathsf{C}(S)$,那么$g\circ f\in\mathrm{End}_\mathsf{C}(S)$。
\end{definition}

\begin{definition}{子范畴 subcategory}
	称$\mathsf{C}'$为范畴$\mathsf{C}$的子范畴,如果$\mathrm{Obj}(\mathsf{C'})\sub\mathrm{Obj}(\mathsf{C})$,且对于任意对象$A,B\in\mathsf{C'}$,成立$\mathrm{Hom}_\mathsf{C'}(A,B)\sub\mathrm{Hom}_\mathsf{C}(A,B)$。
\end{definition}

\begin{definition}{满的 full}
	称范畴$\mathsf{C}$的子范畴$\mathsf{C}'$是满的,如果对于任意对象$A,B\in\mathsf{C'}$,成立$\mathrm{Hom}_\mathsf{C'}(A,B)=\mathrm{Hom}_\mathsf{C}(A,B)$。
\end{definition}

\subsection{范畴例子}

\begin{example}
	集合范畴:
	\begin{itemize}
		\item $\mathrm{Obj}(\mathsf{Set})$:所有集合。
		\item $\mathrm{Hom}_{\mathsf{Set}}(A,B)=B^A$
	\end{itemize}
\end{example}
	
\begin{example}
	矩阵范畴:
	\begin{itemize}
		\item $\mathrm{Obj}(\mathsf{V})=\N$
		\item $\mathrm{Hom}_{\mathsf{V}}(m,n)=\{ \{a_{ij}\}_{m\times n}:a_{ij}\in\R \}$
	\end{itemize}
\end{example}

\begin{example}
	余范畴(opposite category):
	\begin{itemize}
		\item $\mathrm{Obj}(\mathsf{C}^{\mathrm{op}})=\mathrm{Obj}(\mathsf{C})$
		\item $\mathrm{Hom}_{\mathsf{C}^{\mathrm{op}}}(A,B)=\mathrm{Hom}_{\mathsf{C}}(B,A)$
	\end{itemize}
\end{example}

\begin{example}
	点范畴:令$*\in\mathsf{Set}$,定义如下由点$*$诱导的集合范畴$\mathsf{Set}^*$。
	\begin{itemize}
		\item 对象:$\mathrm{Obj}(\mathsf{Set}^*)=(S,s)$,意指态射$\mathrm{Hom}_\mathsf{C}(\{*\},S)$将$*$映为$s\in S$。
		\item 态射:
		$$
		\mathrm{Hom}_{\mathsf{Set}^*}((S,s),(T,t))=\{ \sigma:S\to T\mid \sigma(s)=t \}
		$$
	\end{itemize}
\end{example}

\begin{example}
	关系范畴:令$S$为一个集合,关系$\sim$满足自反性和传递性,定义如下由$S$以及$\sim$诱导的关系范畴$\mathsf{C}_{S,\sim}$。
	\begin{itemize}
		\item $\mathrm{Obj}(\mathsf{C}_{S,\sim})=S$
		\item $$
		\mathrm{Hom}_{\mathsf{C}_{S,\sim}}(a,b)=\begin{cases}
			\{(a,b)\},\quad & a\sim b\\
			\varnothing,\quad & a\not\sim b
		\end{cases}
		$$
	\end{itemize}
\end{example}

\begin{example}
	切片范畴(slice category)和余切范畴(co-slice category):对于范畴$\mathsf{C}$,以及对象$S\in\mathrm{Obj}(\mathsf{C})$,定义如下由$S$诱导的切片范畴$\mathsf{C}_S$。
	\begin{itemize}
		\item 对象:
		$$
		\mathrm{Obj}(\mathsf{C}_S)=\{ (f,A):f\in\mathrm{Hom}_\mathsf{C}(A,S),A\in\mathrm{Obj}(\mathsf{C}) \}
		$$
		通常以如下图表示
		$$
		\xymatrix{
			A\ar_f[d]\\
			S
		}
		$$
		\item 态射:
		$$
		\mathrm{Hom}_{\mathsf{C}_A}((f,A),(g,B))=\{ \sigma\in\mathrm{Hom}_\mathsf{C}(A,B):f=g\circ\sigma \}
		$$
		通常以如下交换图表示
		$$
		\xymatrix{
			A \ar[rr]^\sigma \ar[dr]_{f} & & B \ar[dl]^{g}\\
			&S
		}
		$$
	\end{itemize}
	对于范畴$\mathsf{C}$,以及对象$S\in\mathrm{Obj}(\mathsf{C})$,定义如下由$S$诱导的余切范畴$\mathsf{C}^S$。
	\begin{itemize}
		\item 对象:
		$$
		\mathrm{Obj}(\mathsf{C}_S)=\{ (f,A):f\in\mathrm{Hom}_\mathsf{C}(S,A),A\in\mathrm{Obj}(\mathsf{C}) \}
		$$
		通常以如下图表示
		$$
		\xymatrix{
			S\ar_f[d]\\
			A
		}
		$$
		\item 态射:
		$$
		\mathrm{Hom}_{\mathsf{C}_A}((f,A),(g,B))=\{ \sigma\in\mathrm{Hom}_\mathsf{C}(A,B):g=\sigma\circ f \}
		$$
		通常以如下交换图表示
		$$
		\xymatrix{
			&S\ar[dl]_{f} \ar[dr]^{g}&\\
			A \ar[rr]_\sigma  & & B 
		}
		$$
	\end{itemize}
\end{example}

\begin{example}
	对于对象$A,B\in\mathsf{C}$,定义范畴$\mathsf{C}_{A,B}$如下。
	$$
	\mathrm{Obj}:\qquad
	\xymatrix{
		& A \\
		Z\ar[ur]^f \ar[dr]_g \\
		& B
	}
	\qquad\qquad 
	\mathrm{Hom}:\qquad
	\xymatrix{
		& &A \\
		Z_1 \ar[r]^{\sigma} \ar@/^/[urr]^{f_1} \ar@/_/[drr]_{g_1} & Z_2 \ar[ur]^{f_2} \ar[dr]_{g_2} \\
		& &B
	}
	$$
	
	对于对象$A,B\in\mathsf{C}$,定义范畴$\mathsf{C}^{A,B}$如下。
	$$
	\mathrm{Obj}:\qquad
	\xymatrix{
		A \ar[dr]^f&\\
		&Z  \\
		B \ar[ur]_g&
	}
	\qquad\qquad 
	\mathrm{Hom}:\qquad
	\xymatrix{
		A \ar[dr]^{f_1} \ar@/^/[drr]^{f_2} & & \\
		& Z_1 \ar[r]^\sigma & Z_2 \\
		B \ar[ur]_{g_1} \ar@/_/[urr]_{g_2} & & \\
	}
	$$
\end{example}

\begin{example}
	对于态射$\alpha:A\to C,\beta:B\to C$,定义范畴$\mathsf{C}_{\alpha,\beta}$如下。
	$$
	\mathrm{Obj}:\qquad
	\xymatrix{
		& A \ar[dr]^\alpha \\
		Z\ar[ur]^f \ar[dr]_g & & C\\
		& B \ar[ur]_\beta
	}
	\qquad 
	\mathrm{Hom}:\qquad
	\xymatrix{
		& &A \ar[dr]^\alpha\\
		Z_1 \ar[r]^{\sigma} \ar@/^/[urr]^{f_1} \ar@/_/[drr]_{g_1} & Z_2 \ar[ur]^{f_2} \ar[dr]_{g_2} & & C\\
		& &B \ar[ur]_\beta
	}
	$$
	
	对于态射$\alpha:C\to A,\beta:C\to B$,定义范畴$\mathsf{C}^{\alpha,\beta}$​如下。
	$$
	\mathrm{Obj}:\qquad
	\xymatrix{
		& A \ar[dr]^f \\
		C\ar[ur]^\alpha \ar[dr]_\beta & & Z\\
		& B \ar[ur]_g
	}
	\qquad 
	\mathrm{Hom}:\qquad
	\xymatrix{
		& A \ar[dr]^{f_1} \ar@/^/[rrd]^{f_2} & & \\
		C \ar[ur]^\alpha \ar[dr]_\beta & & Z_1 \ar[r]^\sigma & Z_2 \\
		& B \ar[ur]_{g_1} \ar@/_/[rru]_{g_2} & & \\
	}
	$$
\end{example}

\section{态射}

\subsection{同构}

\begin{definition}{左逆态射 left-inverse morphism}
	对于范畴$\mathsf{C}$,以及对象$A,B\in\mathrm{Obj}(\mathsf{C})$,称态射$g\in\mathrm{Hom}_\mathsf{C}(B,A)$为态射$f\in\mathrm{Hom}_\mathsf{C}(A,B)$的左逆态射,如果成立$g\circ f=\mathbbm{1}_A$。
\end{definition}

\begin{definition}{右逆态射 right-inverse morphism}
	对于范畴$\mathsf{C}$,以及对象$A,B\in\mathrm{Obj}(\mathsf{C})$,称态射$g\in\mathrm{Hom}_\mathsf{C}(B,A)$为态射$f\in\mathrm{Hom}_\mathsf{C}(A,B)$的右逆态射,如果成立$f\circ g=\mathbbm{1}_A$。
\end{definition}

\begin{definition}{同构态射 isomorphism}
	对于范畴$\mathsf{C}$,以及对象$A,B\in\mathrm{Obj}(\mathsf{C})$,称态射$f\in\mathrm{Hom}_\mathsf{C}(A,B)$为同构态射,如果存在逆态射$g\in\mathrm{Hom}_\mathsf{C}(B,A)$,使得成立
	$$
	g\circ f=\mathbbm{1}_A,\qquad f\circ g=\mathbbm{1}_B
	$$
\end{definition}

\begin{definition}{同构的 isomorphic}
	对于范畴$\mathsf{C}$,称对象$A,B\in\mathrm{Obj}(\mathsf{C})$是同构的,且记作$A\cong B$,如果存在同构态射$f\in\mathrm{Hom}_\mathsf{C}(A,B)$。
\end{definition}

\begin{definition}{自同构态射 automorphism}
	对于范畴$\mathsf{C}$,称同构态射$f\in\mathrm{Hom}_\mathsf{C}(S,S)$为对象$S\in\mathrm{Obj}(\mathsf{C})$的自同构态射。
\end{definition}

\begin{definition}{自同构态射群 automorphism group}
	对于范畴$\mathsf{C}$,对象$S\in\mathrm{Obj}(\mathsf{C})$的自同构态射构成自同构态射群$\mathrm{Aut}_\mathsf{C}(S)$。
\end{definition}

\begin{proposition}{同构态射的性质}
	\begin{itemize}
		\item 同构态射存在且存在唯一逆。
		\item 恒等态射$\mathbbm{1}$为同构态射,且$\mathbbm{1}^{-1}=\mathbbm{1}$。
		\item 同构态射$f$的逆$f^{-1}$亦为同构态射,且$(f^{-1})^{-1}=f$。
		\item 如果$f,g$均为同构态射,那么$g\circ f$亦为同构态射,且$(g\circ f)^{-1}=f^{-1}\circ g^{-1}$。
	\end{itemize}
\end{proposition}

\subsection{单态射与满态射}

\begin{definition}{单态射 monomorphism}
	对于范畴$\mathsf{C}$,以及对象$A,B\in\mathrm{Obj}(\mathsf{C})$,称态射$f\in\mathrm{Hom}_\mathsf{C}(A,B)$为单态射,如果对于任意对象$Z\in\mathrm{Obj}(\mathsf{C})$,以及任意态射$\alpha_1,\alpha\in\mathrm{Hom}_\mathsf{C}(Z,A)$,成立
	$$
	f\circ \alpha_1=f\circ \alpha_2
	\implies
	\alpha_1=\alpha_2
	$$
\end{definition}

\begin{definition}{满态射 epimorphism}
	对于范畴$\mathsf{C}$,以及对象$A,B\in\mathrm{Obj}(\mathsf{C})$,称态射$f\in\mathrm{Hom}_\mathsf{C}(A,B)$为满态射,如果对于任意对象$Z\in\mathrm{Obj}(\mathsf{C})$,以及任意态射$\beta_1,\beta\in\mathrm{Hom}_\mathsf{C}(B,Z)$,成立
	$$
	\beta_1\circ f=\beta_2\circ f
	\implies
	\beta_1=\beta_2
	$$
\end{definition}

\begin{proposition}{存在左逆$\implies$单态射}{存在左逆则为单态射}
	对于范畴$\mathsf{C}$,以及对象$A,B\in\mathrm{Obj}(\mathsf{C})$,如果态射$f\in\mathrm{Hom}_\mathsf{C}(A,B)$存在左逆态射,那么态射$f\in\mathrm{Hom}_\mathsf{C}(A,B)$为单态射。
\end{proposition}

\begin{proof}
	如果态射$f\in\mathrm{Hom}_\mathsf{C}(A,B)$存在左逆态射$g\in\mathrm{Hom}_\mathsf{C}(B,A)$,那么任取$\alpha_1,\alpha_2$,满足$f\circ \alpha_1=f\circ \alpha_2$,由于
	$$
	\alpha_1=\mathbbm{1}\circ \alpha_1=g\circ f\circ \alpha_1=g\circ f\circ \alpha_2=\mathbbm{1}\circ \alpha_2=\alpha_2
	$$
	于是$f$是单态射。
\end{proof}

\begin{proposition}{存在右逆$\implies$满态射}
	对于范畴$\mathsf{C}$,以及对象$A,B\in\mathrm{Obj}(\mathsf{C})$,如果态射$f\in\mathrm{Hom}_\mathsf{C}(A,B)$存在右逆态射,那么态射$f\in\mathrm{Hom}_\mathsf{C}(A,B)$为满态射。
\end{proposition}

\begin{proof}
	如果态射$f\in\mathrm{Hom}_\mathsf{C}(A,B)$存在右逆态射$g\in\mathrm{Hom}_\mathsf{C}(B,A)$,那么任取$\beta_1,\beta_2$,满足$\beta_1\circ f=\beta_2\circ f$,由于
	$$
	\beta_1=\beta_1\circ \mathbbm{1}=\beta_1\circ f\circ g=\beta_2\circ f\circ g=\beta_2\circ \mathbbm{1}=\beta_2
	$$
	于是$f$是满态射。
\end{proof}

\begin{proposition}
	单态射的复合为单态射,满态射的复合为满态射。
\end{proposition}

\section{万有性质}

\subsection{初始对象与终止对象}

\begin{definition}{初始对象 initial object}
	对于范畴$\mathsf{C}$,称对象$I\in\mathrm{Obj}(\mathsf{C})$为$\mathsf{C}$的初始对象,如果对于任意对象$S\in\mathrm{Obj}(\mathsf{C})$,存在且存在唯一$\mathsf{C}$中的态射$I\to S$。
	$$
	\xymatrix{
		I\ar[r]^{\exists!\varphi}&\forall S
	}
	$$
\end{definition}

\begin{definition}{终止对象 final object}
	对于范畴$\mathsf{C}$,称对象$F\in\mathrm{Obj}(\mathsf{C})$为$\mathsf{C}$的终结对象,如果对于任意对象$S\in\mathrm{Obj}(\mathsf{C})$,存在且存在唯一$\mathsf{C}$中的态射$S\to F$。
	$$
	\xymatrix{
		S\ar[r]^{\exists!\varphi}&\forall F
	}
	$$
\end{definition}

\begin{definition}{终端对象 terminal object}
	初始对象和终止对象统称为终端对象。
\end{definition}

\begin{proposition}{终端对象的结构}
	对于范畴$\mathsf{C}$,终端对象至多同构。
	\begin{itemize}
		\item 如果$I_1,I_2$均为$\mathsf{C}$的初始对象,那么$I_1\cong I_2$,且同构态射为$I_1\to I_2$。
		\item 如果$F_1,F_2$均为$\mathsf{C}$的终结对象,那么$F_1\cong F_2$,且同构态射为$F_1\to F_2$。
	\end{itemize}
\end{proposition}

\begin{proposition}
	范畴的终止对象为其余范畴的初始对象。
	
	范畴的初始对象为其余范畴的终止对象。
\end{proposition}

\subsection{万有性质}

\begin{definition}{万有性质 universal property}
	称结构(construction)满足万有性质,如果其可被视为一个范畴的终端对象。
\end{definition}

\subsection{商}

\begin{definition}{商范畴 quotient category}
	令$\sim$为集合$S$上的等价关系,定义商范畴$\mathsf{C}_{\sim}^S$如下。
	
	\begin{itemize}
		\item 对象:$\mathrm{Obj}(\mathsf{C}_{\sim}^S)=\{ (\varphi,T)\mid \varphi:S\to T,a\sim b\implies \varphi(a)=\varphi(b) \}$
		\item 态射:$(\varphi,A)\to(\psi,B)$为如下交换图。
		$$
		\xymatrix{
			A \ar^\sigma[rr]& &B\\
			& S \ar[ul]^{\varphi} \ar[ur]_{\psi} &
		}
		$$
	\end{itemize}
\end{definition}

\begin{proposition}{商范畴的初始对象}
	对于商范畴$\mathsf{C}_{\sim}^S$,$(\pi,S/\sim)$为其初始对象,交换图为
	$$
	\xymatrix{
		S/\sim \ar^{\exists!\sigma}[rr]& &\forall T\\
		& S \ar[ul]^{\pi} \ar[ur]_{\forall\varphi} &
	}
	$$
\end{proposition}

\subsection{积}

\begin{proposition}{范畴$\mathsf{C}_{A,B}$的终止对象}
	对于对象$A,B\in\mathsf{C}$,定义范畴$\mathsf{C}_{A,B}$如下。
	$$
	\mathrm{Obj}:\qquad
	\xymatrix{
		& A \\
		Z\ar[ur]^f \ar[dr]_g \\
		& B
	}
	\qquad\qquad 
	\mathrm{Hom}:\qquad
	\xymatrix{
		& &A \\
		Z_1 \ar[r]^{\sigma} \ar@/^/[urr]^{f_1} \ar@/_/[drr]_{g_1} & Z_2 \ar[ur]^{f_2} \ar[dr]_{g_2} \\
		& &B
	}
	$$
	那么其终止对象为
	$$
	\xymatrix{
		& A \\
		A\times B\ar[ur]^{\pi_A} \ar[dr]_{\pi_B} \\
		& B
	}
	$$
\end{proposition}

\begin{definition}{积 product}
	称范畴$\mathsf{C}$存在积,如果对于任意对象$A,B\in\mathrm{Obj}(\mathsf{C})$,范畴$\mathsf{C}_{A,B}$存在终止对象。称该终止对象为$\mathsf{C}$关于$A$和$B$的积,记为$A\times B$,以及$A\times B\to A$和$A\times B\to B$。
\end{definition}

\begin{example}
	\begin{itemize}
		\item 集合范畴$\mathsf{Set}$中,$A\times B$表示$A$与$B$的Descartes积。
		\item 关系范畴$\mathsf{C}_{\R,\le}$中,$a\times b$表示$a$与$b$的最小值$\min(a,b)$。
		\item 群范畴$\mathsf{Grp}$中,$(G,*_G)\times (H,*_H)$表示直积$(G\times H,*_{G\times H})$。
		\item Abel群范畴$\mathsf{Ab}$中,$(G,*_G)\times (H,*_H)$表示直积$(G\times H,*_{G\times H})$。
	\end{itemize}
\end{example}

\subsection{余积}

\begin{proposition}{范畴$\mathsf{C}^{A,B}$的初始对象}
	对于对象$A,B\in\mathsf{C}$,定义范畴$\mathsf{C}^{A,B}$​如下。
	$$
	\mathrm{Obj}:\qquad
	\xymatrix{
		A \ar[dr]^f&\\
		&Z  \\
		B \ar[ur]_g&
	}
	\qquad\qquad 
	\mathrm{Hom}:\qquad
	\xymatrix{
		A \ar[dr]^{f_1} \ar@/^/[drr]^{f_2} & & \\
		& Z_1 \ar[r]^\sigma & Z_2 \\
		B \ar[ur]_{g_1} \ar@/_/[urr]_{g_2} & & \\
	}
	$$
	那么其初始对象为
	$$
	\xymatrix{
		A \ar[dr]^{i_A}&\\
		&A\sqcup B  \\
		B \ar[ur]_{i_B}&
	}
	$$
\end{proposition}

\begin{definition}{余积 coproduct}
	称范畴$\mathsf{C}$存在余积,如果对于任意对象$A,B\in\mathrm{Obj}(\mathsf{C})$,范畴$\mathsf{C}^{A,B}$存在初始对象。称该初始对象为$\mathsf{C}$关于$A$和$B$的余积,记为$A\sqcup B$,以及$A\sqcup B\to A$和$A\sqcup B\to B$。
\end{definition}

\begin{example}
	\begin{itemize}
		\item 集合范畴$\mathsf{Set}$中,$A\sqcup  B$表示$A$与$B$的不交并。
		\item 关系范畴$\mathsf{C}_{\R,\le}$中,$a\sqcup b$表示$a$与$b$的最大值$\max(a,b)$。
		\item 群范畴$\mathsf{Grp}$中,$(G,*_G)\sqcup (H,*_H)$表示自由积$G*H$。
		\item Abel群范畴$\mathsf{Ab}$中,$(G,*_G)\sqcup (H,*_H)$表示为直和$G\oplus H$。
	\end{itemize}
\end{example}

% \end{document}
